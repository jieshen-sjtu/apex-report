\documentclass[12pt,a4paper,twoside]{article}
\usepackage[utf8]{inputenc}
\usepackage{amsmath}
\usepackage{amsfonts}
\usepackage{amssymb}
\usepackage[left=4.00cm, right=4.00cm, top=5.00cm, bottom=5.00cm]{geometry}
\author{Shen Jie}
\title{Summary of Structured Learning}
\begin{document}
\maketitle
\section{Abstract}
\begin{itemize}
\item Focus: deal with \emph{complex outputs}, \emph{i.e.}, trees, graphs, instead of
powerful input representation
\item How to accomplish this?\\
\quad\\
$\Longrightarrow$ maximum-margin formulation(SVM)
\end{itemize}
\section{Introduction}
\begin{itemize}
\item Task: learning a mapping from input vectors $x \in \mathcal{X}$ to
response variables $y \in \mathcal{Y}$.\\
\quad\\
$\Longrightarrow$ Note: $y$ can be a scalar or \emph{structured object}
\item Approach: specifying discriminant functions to exploit
the structure and dependencies with $\mathcal{Y}$, meanwhile
provide an efficient algorithm to deal with very large output
spaces.\\
\quad\\
$\Longrightarrow$ Key: dealing with more complex output spaces by
extracting \emph{combined features} rather than define more complex
functions
\end{itemize}

\section{Large Margin Learning with Joint Feature Map}
\begin{itemize}
\item Primal problem: learning functions $f:\mathcal{X}\rightarrow
\mathcal{Y}$ based on training samples.
\item Transferred problem: learn a \emph{discriminant function}
\begin{eqnarray*}
F:\mathcal{X}\times \mathcal{Y} \rightarrow \mathbb{R}\\
(x,y)\mapsto r
\end{eqnarray*}
where a high value of $r$ indicates the hypothesis $y$ is more
possible to be the actual response.
\item Suppose we have the discriminant function, then turn back
to the primal problem, $f$ is
\begin{eqnarray}
f(\mathbf{x})=\arg\max_{\mathbf{y}\in \mathcal{Y}}F(\mathbf{x,y}),\\
f(\mathbf{x;w})=\arg\max_{\mathbf{y}\in \mathcal{Y}}F(\mathbf{x,y;w})
\end{eqnarray}
\label{eqn1}
\item Simplification: assume $F$ be linear in some \emph{combined feature}
representation
\[
F(\mathbf{x,y;w})=\langle \mathbf{w},\Psi(\mathbf{x,y})\rangle
\]
Challenges:
\begin{enumerate}
\item how to design $\Psi$
\item how to train \textbf{w}
\item how to solve the optimal problem(\emph{Equation 2})
\end{enumerate}
Notes:
\begin{enumerate}
\item the dimension of $\Psi$ should be consistent for all
pairs $\mathbf{(x,y)}$, since we only train a unique parameter
vector $\mathbf{w}$
\item The specific form of $\Psi$ depends on the nature of the problem
\item The training of \textbf{w} follows a general SVM paradigm(discussed
in the following paragraph)
\item $f(\mathbf{x;w})$ can be efficiently computed via the CKY algorithm
\end{enumerate}
\end{itemize}

\subsection{Loss Functions and Risk Minimization}
This subsection seems irrelevant with the following contexts...
\subsection{Margin Maximization}
\subsubsection{SEPARABLE CASE}
Zero training error:
\[
\forall i \in \{1,...,n\}:\quad \max_{\mathbf{y}\in\mathcal{Y}\backslash\mathbf{y}_i}
\{\langle \mathbf{w},\Psi(\mathbf{x}_i,\mathbf{y})\rangle\}\leq
\langle \mathbf{w},\Psi(\mathbf{x}_i,\mathbf{y}_i)\rangle
\]
Rewrite it as a convex quadratic program in standard form
\begin{eqnarray}
SVM_0:\qquad \min_{\mathbf{w}}\frac{1}{2}||\mathbf{w}||^2\\
s.t.\quad \forall i, \forall \mathbf{y}\in \mathcal{Y}:
\langle\mathbf{w},\delta \Psi(\mathbf{y})\rangle \geq 1
\end{eqnarray}
\end{document}